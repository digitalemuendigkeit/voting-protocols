\chapter{Fazit}
%\markboth{Formelzeichen und Symbole}{}
In dieser Arbeit wurde ein agent*innenbasiertes Modell zur Simulation einer Social Media Plattform entwickelt und in Julia implementiert. Auf der modellierten Plattform können User*innen mit Posts interagieren, indem sie diese bewerten. Das Nutzer*innenverhalten wird für Plattformen, auf denen Konsens und Dissens herrscht, modelliert. Es werden vier Bewertungsmetriken, darunter die Reddit Hot und die Hacker News Metrik, eingeführt. Für die Bewertungsmetriken wurde ein Fairnessbegriff definiert, anhand dessen sie vergleichbar werden. Die Bewertungsmetriken besitzen einige variable Parameter, wie den Initialscore, die Zufallsabweichung und die Bewertungstransformation.  Mit einem in Julia entwickelten Framework können die Bewertungsmetriken und das agent*innenbasierte Modell konfiguriert und simuliert werden. Die Bewertungsmetriken werden mit Evaluationsparametern wie dem $nDCG$ oder dem Gini-Koeffzienten ausgewertet. Die Ergebnisse der Simulation zeigen, dass die Bewertungsmetriken sich in ihrer Fairness unterscheiden und die Wahl der fairsten Bewertungsmetrik von der Art der Plattform abhängig ist. Die artifiziell erzeugten User*innenparamter besitzen nur einen geringen Einfluss auf die Simulationsergebnisse


\begingroup
\renewcommand{\cleardoublepage}{}
\renewcommand{\clearpage}{}
\chapter{Ausblick}
\endgroup

Die Simulationen wurden mit kleinen Iterationlängen, User*innen- und Postanzahlen durchgeführt. Einige Bewertungsmetriken werden durch eine höhere Iterationszahl besser, während andere im Ergebnis stabil bleiben. Wird die Iterationslänge, wie die User*innen- und Postanzahlen erhöht, werden die Ergebnisse genauer und ermöglichen besseren Aufschluss, welche Bewertungsmetrik am fairsten ist. Es wäre interessant Simulationen mit höheren Anzahlen durchzuführen.

Die Implementierung ist beschränkt auf Votingsysteme mit  $V_N = \{1,2\}$, Modelle, die User*innen Posts nur positiv oder positiv und negativ bewerten können. Es zeigt sich, dass der zweidimensionale Bewertungsraum in mehr Konfigurationen bessere Ergebnisse liefert. Es wäre interessant, zu untersuchen wie sich Modelle mit größeren $V_N > 2$ verhalten.

Im Modell sind die Verteilungen, welche das User*innenverhalten definieren nicht korreliert. Dies ist in der Realität jedoch sicherlich der Fall. Nutzer*innen, die häufig aktiv sind, weisen sicherlich auch ein ausgeprägteres Bewertungsverhalten aus, sie bewerten Posts häufiger. Eine zukünftige Arbeit könnte die User*innenparameter in Korrelation setzen.

Wenn User*innen im Modell einen Post betrachten, entscheiden sie rein nach der wahrgenommenen Qualität, ob und wie sie den Post bewerten. Der soziale Einfluss, der in einem realem System durch die angezeigten Bewertungen entsteht, wird in diesem Modell nicht erfasst. Um dies zu realisieren, muss die Bewertungszufriedenheit durch den neuen sozialen Einfluss angepasst werden. Auch das Alter der Posts wird in den aktuellen User*innenmeinungsfunktionen nicht wahrgenommen, die User*innenmeinungsfunktionen könnten dahingehend angepasst werden.


Die Framework bietet die einfache Definition von neuen Bewertungsmetriken und deren Parameter. In \cite{dietze} wird das \textit{Dirichlet Smoothing} verwendet, um die Bewertungen der Posts zu transformieren, dies könnte ebenfalls implementiert und getestet werden. Aktuell können nur fixe Initialscores für Posts angegeben werden, es ist jedoch auch denkbar die Initialscores dynamisch zu berechnen. Neue Posts könnten zum Beispiel immer den Mittelwert der aktuellen Scores der Posts als Initialscore erhalten.

Der Featureraum für die Parameter der Bewertungsmetrik und des Votingsystems kann begrenzt werden, sodass ein Optimierungsproblem formuliert werden und zum Beispiel nach der Minimierung von $\rho$ optimiert werden kann. So wäre es möglich für eine bestimmte Plattformart die optimale Bewertungsmetrik zu finden.