\chapter{User*innenmeinung}
\label{chapter:bewertungsfunktionen}


Für eine User*in $U$ die einen Post $P$ betrachtet ist die Meinungsfunktion $R(P,U): \R^{N_Q} \rightarrow \R$, welche die beiden $N_Q$-dimensionalen Vektoren der Postqualität und Qualitätsperzeption auf den skalaren Meinungswert $m_{P,U} = R(P, U)$ reduziert. Dieser drückt aus wie gut die User*in den betrachteten Post empfindet. Dabei soll gelten:

\begin{enumerate}
	\item $m_{P,U} \in [0,1]$
	\item Bei $m_{P,U} = 1$ wird der Post von der User*in als maximal gut empfunden
	\item Bei $m_{P,U} = 0$ wird der Post von der User*in als maximal schlecht empfunden
\end{enumerate}

\section{Transformation der Qualitätsparameter}

Um die Intervallgrenzen der genannten Kriterien zu beachten wird eine Transormation der Qualitätsparameter vollzogen, welche diese auf das Intervall $[0,1]$ begrenzen. Die Transformation wird durch die logistische Funktion $l$ in Formel \ref{sigmoid} vollzogen:



\begin{equation}
\label{sigmoid}
l(x) = \frac{1}{1 + e^{-\frac{1}{2}x}}
\end{equation}

%TODO grafik der logistischen funktion

Somit ergeben sich die transformierten Qualitäts- bwz Qualitätsperzeptionsvektoren der Post und User*innen durch die komponentenweise Anwendung von $s$ auf $q_P$ und $q_U$

\begin{align}
\tilde{q}_P = l(q_P) \\
\tilde{q}_U = l(q_U) 
\end{align}

\section{Approxamtion des User*innenratings}

Im folgenden werden zwei Ansätze eingeführt um die User*innenmeinungsfunktion $R(P,U)$ zu approximieren.
 
\subsection{Konsensrating}

Im Konsensrating wird davon ausgegangen, dass User*innen "gute" Posts eher als diese erkennen. Alle User*innen sind sich über die besonders guten und schlechten Posts einig und erkennen diese richtig. Bei durchschnittlichen Posts kann es auch zu Meinungsverschiedenheiten kommen, je nach der Qualitätsperzeption der User*innen.

Das Konsensrating ist eher auf einen Q\&A Platform oder einem Techforum wie Hacker News denkbar. Konstruktive Beiträge sind eher von destruktiven zu unterscheiden. User*innen die besonders gut in einem Thema sind haben an Beiträge zu diesem Thema möglicherweise höhere Ansprüche.

In Formel \ref{konsensexp2} ist ein Ratingfunktion, welche die genannten Punkte ausdrückt:

\begin{equation}
\label{konsensexp2}
R_K(P,U) = \frac{1}{N}\sum_{i = 1}^{N}\tilde{q}_{P,i}^{\tilde{q}_{U,i}}
\end{equation}



\subsection{Dissensrating}

User*innen empfinden Posts als "gut", die nah an ihrer eigenen Qualitätsperzeption liegen. Dadurch sind sich User*innen bei vielen Posts uneinig.

Das Dissensrating ist eher auf einer Diskussionsplattformen denkbar. Dort können die Meinungen auseinander gehen. User*innen haben nicht unbedingt das Interesse die Position der Gegenüber einzunehmen. 

Eine Funktion die die genannten Kriterien erfüllt verwendet die euklidische Distanz, es ergibt sich in Formel \ref{dissens}:


%TODO Dissensrating validieren
\begin{equation}
\label{dissens}
R_D(P,U) = 1 - \frac{||(\tilde{q}_P - \tilde{q}_U)||_2}{\sqrt{N_Q}}
\end{equation}



