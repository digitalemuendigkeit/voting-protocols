\chapter{User*innenmeinung}
\label{chapter:bewertungsfunktionen}


Für eine User*in $U$ die einen Post $P$ betrachtet ist die Meinungsfunktion $R(P,U): \R^{N_Q} \rightarrow \R$, welche die beiden $N_Q$-dimensionalen Vektoren der Postqualität und Qualitätsperzeption auf den skalaren Meinungswert $m_{P,U} = R(P,U)$ reduziert. Dieser drückt aus wie gut die User*in den betrachteten Post empfindet. Dabei soll gelten:

\begin{enumerate}
	\item $m_{P,U} \in [0,1]$
	\item Bei $m_{P,U} = 1$ wird der Post von der User*in als maximal gut empfunden
	\item Bei $m_{P,U} = 0$ wird der Post von der User*in als maximal schlecht empfunden
\end{enumerate}

\section{Transformation der Qualitätsparameter}

Um die Intervallgrenzen der genannten Kriterien zu erfüllen wird eine Transformation der Qualitätsparameter vollzogen, welche diese auf das Intervall $[0,1]$ begrenzen. Die Transformation wird durch die logistische Funktion $l$ in Formel \ref{sigmoid} vollzogen:



\begin{equation}
\label{sigmoid}
l(x) = \frac{1}{1 + e^{-\frac{1}{2}x}}
\end{equation}

%TODO grafik der logistischen funktion, NEE EHER NICHT

Somit ergeben sich die transformierten Qualitäts- bwz Qualitätsperzeptionsvektoren der Post und User*innen durch die komponentenweise Anwendung von $l$ auf $q_P$ und $q_U$

\begin{align}
\tilde{q}_P = l(q_P) \\
\tilde{q}_U = l(q_U) 
\end{align}

\section{Approxamtion der User*innenmeinungfunktion}


Die reale User*innenbewertungsfunktion ist hoch komplex und kann nur sehr vereinfacht modelliert werden.
Im folgenden werden zwei Ansätze eingeführt um die User*innenmeinungsfunktion $R(P,U)$ zu approximieren.
 
\subsection{Meinung im Konsens}

In der Konsensbewertung wird davon ausgegangen, dass alle User*innen "gute" Posts mit hohen Qualitätsmerkmalwerten eher als gut empfinden und Posts mit niedrigen Qualitätsmerkmalwerten eher als schlecht empfunden werden.

Die Konsensbewertung ist auf technischen und wissenschaftlichen Plattformen, zum Beispiel Hacker News, denkbar. Konstruktive Beiträge sind eher von destruktiven zu unterscheiden. Trotzdem können User*innen unterschiedlich viel Wissen in einzelnen Fachgebieten und dadurch einen höheren Anspruch an Posts in diesem Bereich haben.  

In Formel \ref{konsensexp2} wird die Konsensbewertung ausgedrückt: 

%TODO Für Q_N einsetzen
\begin{equation}
\label{konsensexp2}
R_K(P,U) = \frac{1}{N}\sum_{i = 1}^{N}\tilde{q}_{P,i}^{\tilde{q}_{U,i}}
\end{equation}

Der transformierte Qualitätsvektor $\tilde{q}_P$ des Posts wird komponentenweise mit der transformierten Qualitätsperzeption der User*in $\tilde{q}_U$ exponiert und aufsummiert. Die Summe wird durch die Anzahl der Qualitätsmerkmale $Q_N$ geteilt.


\subsection{Meinung im Dissens}

User*innen empfinden Posts als "gut", die nah an ihrer eigenen Qualitätsperzeption liegen. Dadurch sind sich User*innen bei vielen Posts uneinig.


Die Dissensbewertung ist auf einer Diskussionsplattform denkbar. Dort können die Meinungen über Beiträge stark auseinander. User*innen wollen Beiträge, welche ihre eigene Meinung wiedergeben für andere User*innen sichtbar machen.



In Formel \ref{dissens} wird die Dissensbewertung mithilfe der euklidischen Distanz beschrieben. Es wird die euklidische Distanz aus $\tilde{q}_P$ und $\tilde{q}_U$ gebildet und durch $\sqrt{N_Q}$ geteilt. Dieser Term wird von $1$ abgezogen. Eine User*in für $\tilde{q}_P = \tilde{q}_U$ einen Post maximal gut.


%TODO Dissensrating validieren
\begin{equation}
\label{dissens}
R_D(P,U) = 1 - \frac{||(\tilde{q}_P - \tilde{q}_U)||_2}{\sqrt{N_Q}}
\end{equation}

In Abbildung \ref{fig:vergleichmeinung} werden anhand von jeweils zwei Beispielposts ,$P1$,$P2$, und User*innen, $U1$,$U2$,die beiden Meinungsfunktionen mit zwei Qualitätsmerkmalen $x$ und $y$ verglichen. Die Posts werden nach Farbe unterschieden, während die User*innen nach dem Symbol unterschieden. In der linken Abbildung sind die User*innen nach ihren transformierten Qualitätsmerkmalen dargstellt. Rechts ist der Meinungswert aus den beiden Meinungsfunktionen dargestellt. Herrscht Dissens sind sich beide User*innen uneinig welchen Post sie besser finden. Im Konsens sind sich die User*innen einig, dass $P2$ besser ist. Trotzdem empfindet $U2$ den Post $P1$ deutlich schlechter als $U1$.

\begin{figure}
	\begin{subfigure}{0.6\textwidth}
		\includegraphics[width=\textwidth]{"../plots/meinung1.png"}
	\end{subfigure}
	\begin{subfigure}{0.4\textwidth}
		\includegraphics[width=\textwidth]{"../plots/meinung2.png"}
	\end{subfigure}
	\label{fig:vergleichmeinung}
	\caption{Vergleich der beiden Meinungsfunktionen Konsens und Dissens}
\end{figure}



