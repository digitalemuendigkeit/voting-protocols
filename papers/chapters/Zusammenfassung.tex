\markboth{}{}
\paragraph{Zusammenfassung}
Die riesige Anzahl an Beiträgen auf sozialen Medien ist nicht überschaubar. Die Beiträge müssen vorsortiert werden, sodass den Nutzer*innen möglichst die interessanten und relevanten Beiträge angezeigt werden. Für diese Sortierung werden Bewertungsmetriken verwendet, welche die Posts anhand ihrer Eigentschaften, zum Beispiel der Anzahl und Art der Bewertungen, die sie von den User*innen erhalten haben, sortieren.

In dieser Arbeit wird ein agent*innenbasiertes Modell von Votingsystemen in der Onlinekommunikation entwickelt, mithilfe dessen unterschiedliche Bewertungsmetriken nach einem definierten Fairnessbegriff verglichen werden. Dazu werden einige Bewertungsmetriken definiert und auf unterschiedlichen Arten von Plattformen, wie News- und Q\&A-Plattformen, in einer agent*innenbasierten Simulation eines sozialen Mediums angewendet und ausgewertet. Die getesteten Bewertungsmetriken unterscheiden sich in ihrer Fairness. Die Wahl der Bewertungsmetrik hat somit einen relevanten Einfluss auf das soziale Medium.
	
	