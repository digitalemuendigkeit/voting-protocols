\chapter{Modell von Votingsystemen}
\label{chapter:modell}

In diesem Modell von Votingsystemen können User*innen können Posts veröffentlichen. Diese sind für alle anderen User*innen sichtbar und können bewertet werden. Die Posts werden nach einer Bewertungsmetrik unter Betrachtung der Postparameter, wie die Anzahl und Art der Bewertungen, Betrachtungen und Zeitpunkt der Veröffentlichung bewertet. Auf der Startseite der Plattform werden die Posts absteigend nach ihrer Bewertung sortiert in einer vertikalen Liste angezeigt. Je weiter ein*e User*in auf der Seite herunterscrollt, desto mehr Posts werden angezeigt. Die Posts sind nicht auf Seiten aufgeteilt, sondern befinden sich in einer kontinuierlichen Liste. Die Liste der Posts ist für alle User*innen, die die Plattform zum gleichen Zeitpunkt besuchen identisch und nicht personalisiert.

Im Laufe der Modellierung erhält die Plattform keine neue User*innen, es werden jedoch neue Posts erstellt und hinzugefügt.
Das agent*innenbasierte Modell \ac{M} wird für eine festgelegte Anzahl Iterationsschritte $N_M$ simuliert. Für jeden Iterationsschritt $i_M$ werden bestimmte Aktionen durchgeführt. Die User*innen interagieren in jedem Schritt mit dem Modell.

In diesem Kapitel werden die in der Simulation benötigten Modellparameter beschrieben.

\section{Definition der Qualität}

Posts in sozialen Medien besitzen eine Qualität, welche User*innen wahrnehmen. Die Postqualität besitzt mehrere Merkmale, wie zum Beispiel den Informationsgehalt, die Originalität oder die Richtigkeit der Grammatik und Rechtschreibung. User*innen nehmen die bestimmte Qualitätsmerkmale unterschiedlich wahr. Manchen User*innen erachten den Informationsgehalt eines Posts als wichtig, während andere eher auf die Witzigkeit eines Beitrages achten.

Welche und wie viele Qualitätsmerkmale in der Realität existieren kann nicht ermittelt werden, daher wird die Qualität künstlich durch Vektoren der Dimension $n_Q$ modelliert. Jeder Eintrag eines Qualitätsvektor beschreibt die Ausprägung der Qualität in einem Qualitätsmerkmal. Ein Qualitätsvektor der Dimension $n_Q$ beschreibt kann somit ein mit $n_Q$ Qualitätsmerkmalen beschreiben. 

In einem Modell besitzen sämtliche Qualitätsvektoren die gleiche Dimension. Sie werden über eine kontinuierliche $n_Q$-dimensionale Verteilung $V_Q$ mit dem Mittelwert $\mu = 0$ erzeugt. Es bietet sich die Verwendung einer $n_Q$-dimensionalen Normalverteilung an. Einzelne Qualiätsmerkmale können so einfach in Korrelation gesetzt werden. 

\section{Bewertungraum des Votingsystems}

Ein Votingsystem kann einen Bewertungsraum $B$ mit beliebiger Dimension $n_B$ zur Verfügung stellen um Posts von User*innen bewerten zu lassen. Sind nur Upvotes, wie bei Hacker News, erlaubt, so handelt es sich um einen eindimensionalen Bewertungsraum, Im Reddit Hot Ranking sind auch Downvotes erlaubt, ein Bewertungsraum mit $n_B = 2$. Viele Onlineshops lassen in einem fünfdimensionalem Raum Artikel mit 1 bis 5 Sternen bewerten.

Diese Arbeit ist beschränkt auf $n_B = \{1,2\}$. 

\section{Posts}

Posts besitzen beobachtbare Parameter, welche durch die User*inneninteraktion mit der Plattform entstehen und verändert werden. Diese Parameter sind auf einer produktiven Social Media Plattform üblicherweise in einer Datenbank gespeichert. Die beobachtbaren Parameter können in den Bewertungsmetriken verwendet werden.
Außerdem besitzen Posts nicht beobachtbare Parameter, welche von der Umwelt abhängig sind und artifiziell durch die Modellierung erzeugt werden.

\subsection{Beobachtbare Postparameter}

\paragraph{Veröffentlichungszeitpunkt}
Der Veröffentlichungszeitpunkt $a_P = i_M$ ist der Modellschritt $i_M$, in welchem der Post veröffentlicht wurde.

\paragraph{Bewertungsvektor}

Der Bewertungsvektor $b_P$ eines Posts besitzt die Dimension $n_B$. In ihm wird Bewertung der User*innen des Postes gespeichert.

\paragraph{Score}

Der Score $s_P$ eines Posts enthält den Wert, den der Post durch eine Bewertungsmetrik (Abschnit \ref{chapter:bewertungsmetriken} zugeordnet wurde.

Der Initialscore $s_{0}$ aller Posts kann varriert werden.

\paragraph{Betrachtungen}

Die Anzahl $w_P$ der User*innen  die den Post gesehen haben.

\subsection{Nicht beobachtbare Postparameter}

\paragraph{Qualität}
\label{pqualitaet}

Der $n_Q$-dimensionale Qualitätsvektor $q_P$ beschreibt die Qualität eines Posts. Er wird aus der Qualitätsverteilung $V_Q$ erzeugt. 

Je größer einzelne Einträge in $q_P$ sind, desto besser ist die Qualität eines Postes in diesem Qualitätsmerkmal.

%Es wird keine Unterscheidung zwischen Darstellungsqualität wie in [Felix masterarbeit] gemacht. Diese ist hoch, wenn auf den ersten Blick deutlich wird um was sich der Post handelt. Alle User*innen erfassen auf den ersten Blick den gesamten Qualitätsumfang der Posts.

\paragraph{Relevanz}

Die Relevanz gibt an, wie interessant ein Post ist. Relevante Posts sollen auf einer Kommunikationsplattform weit oben angezeigt werden.

Die Relevanz eines Posts kann sich mit der Zeit ändern. Auf ein Newsplattform spielt Akualität eine wichtige Rolle. Posts welche gerade erst veröffentlicht wurden besitzen meist eine höhere Relevanz als Posts, welche vor langer Zeit veröffentlicht wurden. Auf eine Q\&A-Seite verlieren Beiträge mit versteichender Zeit weniger an Relevanz. Die beste Antwort auf eine Frage bleibt dies meistens auch für längere Zeit.

Die Relevanz $R_P$ eines Posts in Formel \ref{eq:gesqualitaet} ergibt sich somit durch die Summe aller Einträge vo $q_P$ dividiert durch das in Gravität gesetzte Alter des Posts. Durch die Gravität $G_R$ wird festgelegt wie schnell die Posts an Relevanz verlieren:

\begin{equation}
\label{eq:gesqualitaet}
{R_P} = \frac{1}{(i_M - a_P)^{G_R}}\sum_{i=1}^{N_Q} {q_P}_i
\end{equation}

Für eine Q\&A-Seite ist $G_R = 0$ anzunehmen, während für eine Newsseite $G_R = 2$ gewählt werden kann.

\section{Bewertungsmetriken}
\label{sec:bewertungsmetrik}

Bewertungsmetriken sollen Posts anhand ihrer Relevanz sortieren. Die Relevanz ist jedoch ein nicht beobachtbarer Parameter. Zur Approximation der Relevanz können nur beobachtbare Postparameter verwendet werden.

Das Kapitel \ref{chapter:bewertungsmetriken} beschäftigt sich ausführlich mit Bewertungsmetriken.

%\subsection{Vorsortierung der Posts}

%Bei der Simulation einer Kommunikationsplatform besteht die Problematik des \textit{Cold Starts}. Diese besagt, dass es in einer realen Umgebung den Zeitpunkt $0$ der Simulation nicht gibt. Zum Zeitpunkt $0$ der Simulation sind die Posts zufällig angeordnet. Kein Posts wurde betrachtet oder bewertet.

%Um dem \textit{Cold Start} entgegen zu wirken können die Posts nach Qualität vor der Simulation sortiert werden.
%Der Sortierungskoeffizient $\omega$ befindet sich im Intervall $[-1,1]$. Bei $\omega=0$ findet keine Vorsortierung statt. Bei $\omega=1$ sind die Posts vollständig nach Qualität sortiert, bei $\omega=-1$ sind die Posts vollständig umgekehrt nach Qualität sortiert.


\section{User*innen}


Um ein agent*innenbasiertes Modell einer Onlinekommunikationsplattform zu entwickeln ist es elementar das User*innenverhalten zu beschreiben. Die Parameter werden stattdessen durch plausible Verteilungen definiert.

\subsection{Aktivtität}


Die Aktitvität  $a_P$ beschreibt, wie oft User*innen aktiv sind, die Kommunikationsplattform besuchen und Posts betrachten. Dabei sind weder alle User*innen durchgehend noch gleichzeitig online.  

Die Aktivität wird als Wahrscheinlichkeit modelliert. Die Aktivität beschreibt wie wahrscheinlich es ist, dass ein*e User*in in einem Iterationsschritt aktiv ist. Die Aktivität ist über die User*innen $\beta$-verteilt. Einige denkbare $\beta$-Verteilungen sind in Abbildung \ref{fig:betas} zu sehen. Da sie auf das Intervall $[0,1]$ beschränkt sind $\beta$-Verteilungen zur Modellierung von Wahrscheinlichkeiten geeignet. Mit ihnen lässt sich eine große Menge an User*innen beschreiben, welche wenig aktiv sind und eine kleine Menge, welche sehr aktiv ist.

\begin{figure}[!h]
	\includegraphics[width=\textwidth]{"../plots/vote_prob_dists.png}
	\label{fig:betas}
	\caption{Unterschiedliche $\beta$-Verteilungen zur Modellierung der User*innenaktivität}
\end{figure}

%In dem implementierten Modell beschreibt jede Iteration das Verstreichen eines Zeitschrittes (30 Minuten). Wie in \cite{Hogg20121} beschrieben sind nicht alle User*innen zu jedem Zeitpunkt aktiv. Die Aktivität von User*innen lässt sich durch eine logarithmische Normalverteilung
%beschreiben. In diesem Modell wird die Aktivität über eine Wahrscheinlichkeit modelliert,  mit welcher ein*e User*in in einem Modellschritt aktiv ist. Da sich die logarithmische Normalverteilung nicht zufriedenstellend auf das Intervall $ [0,1]$ begrenzt werden kann, wird eine $\beta$-Verteilung zur Approximation verwendet.

\subsection{Konzentration}

Wenn ein*e User*in aktiv ist wird sie nicht alle Posts der Plattform anschauen. Einerseits weil die Plattform dazu zu viele Posts besitzt und andererseits, weil die Konzentration von User*innen auf der Plattform nicht unendlich ist. Wenn User*innen auf der Plattform aktiv sind werden sie stets eine endliche Anzahl an Posts betrachten. Diese Anzahl wird durch die Konzentration definiert.

Die Konzentration $k_P$ ist über die User*innen diskret-positiv verteilt. Es bietet sich die Modellierung mit einer Poisson-Verteilung an, da sich für diese ein diskreter Mittelwert definiert werden kann, um welchen die Konzentrationswerte abweichen.

%TODO [Bild mit unterschiedlichen Poissonverteilungen??]

\subsection{Qualitätsperzeption}
\label{uqualitaet}

Durch die Qualitätsperzeption wird festgelegt wie User*innen Qualität wahrnehmen und wie stark ihre Qualitätswahrnehmung ausgeprägt ist. Die Qualitätsperzeption wird durch einen Vektor $q_U$ mit $n_Q$  Einträgen beschrieben. Jeder Eintrag beschreibt die Qualitätswahrnehmung in dem entsprechenden Qualitätsmerkmal. Je größer ein Eintrag im Qualitätsvektor ist, desto höher ist die Fähigkeit der User*in die wahre Qualität des Posts zu beurteilen.

$q_U$ wird als Zufallswert aus der Qualitätsverteilung erzeugt.

\subsection{Meinungsfunktion}

Wenn eine User*in $U$ einen Post $P$ betrachtet, bildet sie sich über den Post anhand der Postqualität und ihrer eigenen Qualitätsperzeption eine Meinung. Die Meinungsbildung wird durch die User*innenmeinungsfunktion modelliert. Durch diese Funktion $R(P,U): \R^{n_Q} \rightarrow \R $ werden die beiden $n_Q$-dimensionalen Vektoren der Postqualität und Qualitätsperzeption auf einen skalaren Meinungswert $m_{P,U} = R(P,U)$ reduziert. Dieser drückt aus, wie gut die User*in den betrachteten Post empfindet. Je größer der Meinungswert ist, desto besser empfindet die User*in den Post.

Weitere Überlegungen zur User*innenmeinungsfunktion befinden sich Kapitel \ref{chapter:bewertungsfunktionen}.

\subsection{Meinungsverteilung}

Die Verteilung der Meinungswerte $W$ wird zur Hilfe gezogen, um zu entscheiden ob Posts von User*innen bewertet wreden.(Siehe dazu Abschnitt \ref{bewertungszufriedenheit}).

Zur Berechnung der empirischen Verteilungsfunktion werden $n$ Posts und User*innen gebildet. Für jede User*in $U_i$ wird mit jedem Post $P_j$ der Meinungswert $m_{P_j, U_i}$ gebildet. In \ref{eq:meinungsverteilung} ergibt sich die empirische Verteilungsfunktion der verteilten Variable $x$:

\begin{equation}
\label{eq:meinungsverteilung}
F_W(x) = \frac{1}{n^2}\sum_{i,j = 1}^{n} 1_{\{x \leq m_{P_j, U_i}\}}
\end{equation}

Dabei ist  $1_{\{x \leq R(Ps_i,Us_j)\}} = 1$ für $x \leq R(Ps_i,Us_j)$ und sonst $0$.


\subsection{Bewertungszufriedenheit}
\label{bewertungszufriedenheit}
Es wird angenommen, dass User*innen einen Post nur bewerten, wenn sie ausreichend (un)zufrieden mit diesem sind. Der berechnete Meinungswert muss dazu kleiner bzw. größer ein bestimmtes Quantil der empirischen Dichteverteilung von $W$ sein. Dieses $Q_W$-Quantil wird mit der Bewertungszufriedenheit $\beta_U$ eine*r User*in definiert und hängt von der Dimension $N_B$ des Bewertungsraumes ab. Für einen eindimensionalen Bewertungsraum ist $Q_W = 1 - \beta_U$. 

In Abbildung \ref{fig:bewzuf} ist eine beispielhafte Meinungsverteilungsfunktion für $N_B = 1$ in schwarz dargestellt. Für $\beta_U$ ist $Q_W$ auf der $y$-Achse rot aufgetragen. Das $Q_W$-Quantil $x_{Q_W}$ ist ebenfalls in rot auf der $x-Achse$ eingezeichnet. Um eine Postbewertung hervorzurufen muss der Meinungswert des Posts größer als $x_{Q_W}$ sein.

\begin{figure}
	\centering
	\label{fig:bewzuf}
	\includegraphics[width=0.7\textwidth]{../plots/rating_dist.png}
	\caption{Um eine Bewertung einer User*in hervorzurufen, muss der Meingungswert $m_{P,U}$ größer als $x_{Q_W}$ sein.}
\end{figure}

%Es gilt $Q_W = F_W(x_{Q_W})$.

Die Bewertungszufriedenheit ist kontinuierlich auf dem Intervall $[0,1]$ verteilt. Zur Modellierung eignen sich wieder $\beta$-Verteilungen, da so modelliert werden kann, dass viele User*innen wenig Posts und wenige User*innen viele Posts bewerten.



%\subsection{Extreme User*innen}

%Extreme User*innen besitzen eine sehr starke Meinung. Die Qualiätsperzeption ist $q_{U_E} = q_P + E$. Wobei $E$ angibt, wie extrem die Meinung ist. Extreme User*innen sind in jedem Modellschritt aktiv und bewerten jeden Post: $a_{U_E}= \beta_{U_E} = 1$. Außerdem besitzen extreme User*innen eine sehr hohe Konzentration:  $k_{U_E} = 500$.

%$E_M$ ist eine Matrix, Anzahl und Art der extremen User*innen definiert. In Formel \ref{eq:em} ist eine Beispielmatrix gezeigt.

%\begin{equation}
%\label{eq:em}
%{E_M}_{Bsp} = 
%\begin{pmatrix}
%0.02 & 20 \\
%0.05 & -10 \\
%\end{pmatrix}
%\end{equation}

%In der ersten Spalte befindet sich der Anteil an der gesamten User*innenzahl mit dem $E$-Wert in der zweiten Spalte. In diesem Beispiel sind 2\% extreme User*innen mit $E=20$ und 5\% extreme User*innen mit $E = -10$. Die restlichen 93\% User*innen sind "normal".

\section{User*innen und Postanzahl}

In der folgenden Tabelle sind die Modellparameter der User*innen- und Postanzahl angegeben, sämtliche dieser Paramter können variiert werden:

\begin{table}[!htbp]
	\begin{tabularx}{\textwidth}{lX}
		$N_U$&  Anzahl der User*innen\\
		${N_P}_{start}$ & Anzahl der Posts zu Beginn der Simulation \\
		${N_P}_{iter}$ & Anzahl der neuen Posts pro Iteration\\ 
	%	${N_P}$ & Gesamtanzahl der Posts, $N_P = {N_P}_{start} + i_M * {N_P}_{iter}$
	\end{tabularx}
\end{table}
