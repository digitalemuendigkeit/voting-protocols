\chapter{Modell}

\section{Qualitätsraum}

User*innen können Posts nach unterschiedlichen Kriterien und Qualitätsmerkmale betrachten. Auf einer News Platform achten User*innen auf die Informativität und Aktualität. Auf einer Meme-Seite sollen Beiträge vielleicht eher lustig sein. Auf welche  von Beiträgen User*innen achten, lässt sich jedoch nicht bestimmen.

Es wird ein Qualitätsraum $Q$ mit der Dimension $n_Q$ eingeführt. Qualitätsvektoren dieses Raumes können $n_Q$ Qualitätsmerkmale beschreiben.

Qualitätsmerkvektoren werden über eine kontinuierliche $n_Q$-dimensionale Verteilung $V_Q$ mit dem Mittelwert $\mu = 0$ erzeugt. Es bietet sich die Verwendung einer n-dimensionalen Normalverteilung an. Einzelne Qualiätsmerkmale können so einfach in Korrelation gesetzt werden. 

\section{Bewertungraum}

Eine Onlineplaform kann einen Bewertungsraum $B$ mit beliebiger Dimension $n_B$ zur Verfügung stellen um Posts von User*innen bewerten zu lassen. Sind nur Upvotes, wie bei Hacker News, erlaubt, so handelt es sich um einen eindimensionalen Bewertungsraum, Im Reddit Hot Ranking sind auch Downvotes erlaubt, ein Bewertungsraum mit $n_B = 2$. Viele Onlineshops lassen in einem sechsdimensionalem Raum Artikel mit 0 bis 5 Sternen bewerten.

\section{Bewertungsverteilung}

Die Verteilung, die durch die Bewertungsfunktion von User*innen für Posts erzeugt wird ist nicht bekannt und kann nicht analytisch berechnet werden. Es ist jedoch notwendig Quantil der Bewertungsverteilung zu berechnen um zu entscheiden, ob ein*e User*in einen Post bewertet. Die Verteilung wird empirisch berechnet.

Aus der Qualitätsverteilung werden Pseudoposts mit Qualität und Pseudouser*innen mit Qualitätsperzeption erzeugt. Diese werden über ein kartesisches Produkt in die Bewertungsfunktion eingesetzt. Die so berechneten Bewertungswerte bilden die empirische Verteilung. 

%TODO [Vektorrechnung des kartesichen Produktes in die Bewertungsfunktion gestopft]

[Bild mit unterschiedlichen empirischen Bewertungsverteilungen, nach unterschiedlichen Ratingfunktionen]



\section{User*innenparameter}


\subsection{Qualitätsperzeption}
\label{uqualitaet}

Der $n_Q$-dimensionale Qualitätsperzeptionsvektor $q_U$ beschreibt, wie sehr User*innen auf einzelne Qualitätsmerkmale achten. 

Je größer einzelne Einträge von $q_U$, desto mehr achten User*innen auf von diesem Eintrag beschriebene Qualitätsmerkmal von $Q$.

\subsection{Aktivtität}

In dem implementierten Modell beschreibt jede Iteration das Verstreichen eines Zeitschrittes (30 Minuten). Wie in \cite{Hogg20121} beschrieben sind nicht alle User*innen zu jedem Zeitpunkt aktiv. Die Aktivität von User*innen lässt sich durch eine logarithmische Normalverteilung
beschreiben. In diesem Modell wird die Aktivität über eine Wahrscheinlichkeit modelliert,  mit welcher ein*e User*in in einem Modellschritt aktiv ist. Da sich die logarithmische Normalverteilung nicht zufriedenstellend auf das Intervall $ [0,1]$ begrenzt werden kann, wird eine $\beta$-Verteilung zur Approximation verwendet.

[Bild mit unterschiedlichen $\beta$-Verteilungen]

\subsection{Bewertungszufriedenheit}


Über die Bewertungszufriedenheit wird festgelegt, wie (un)zufrieden ein*e User*in mit einem Post sein muss, um diesen zu bewerten. 
Die Bewertungszufriedenheit gibt das Quantil der Bewertungsverteilung an, welches von der Bewertungsfunktion der User*innen für Posts über- bzw. unterschritten werden muss um eine negative bzw. positive Bewertung der User*in für den Post hervorzurufen.

Wie die Aktivitätswahrscheinlichkeit wird Bewertungszufriedenheit von einer $\beta$-Verteilung modelliert.

\subsection{Konzentration}

Die Konzentration gibt an, wie viel Posts ein*e User*in betrachtet, falls sie in einem Modellschritt aktiv ist. User*innen betrachten in jedem Schritt die gleiche Anzahl an Posts.
Sie ist diskret-positiv über die User*innen verteilt. Eine mögliche Verteilung stellt die Poissonverteilung.

[Bild mit unterschiedlichen Poissonverteilungen??]


\section{Postparameter}

%TODO subsections oder paragraphen?
\subsection{Qualität}
\label{pqualitaet}

Der $n_Q$-dimensionale Qualitätsvektor $q_P$ beschreibt die Qualität eines Posts. Er wird aus der Qualitätsverteilung $V_Q$ erzeugt. 

Je größer einzelne Einträge in $q_P$ sind, desto besser wird das durch diesen Eintrag beschriebene Qualitätsmerkmal in $Q$ des Posts angenommen.

\subsection{Bewertungsvektor}

Der Bewertungsvektor $b_P$ eines Posts besitzt die Dimension $n_B$. In ihm wird Bewertung der User*innen des Postes gespeichert.

\subsection{Score}

Der Score $s_P$ eines Posts enthält den Wert, den der Post durch eine Bewertungsmetrik zugeordnet wurde.

\subsection{Veröffentlichungszeitpunkt}

Speichert den Modellschritt $a_P$, indem der Post veröffentlicht wurde

\subsection{Betrachtungen}

Die Anzahl $w_P$ der User*innen  die den Post gesehen haben.
