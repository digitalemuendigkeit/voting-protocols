\chapter{Verwandte Arbeiten}

%Soziale Medien erfreuen sich immer größer User*innenzahlen, die Anzahl der Nutzung steigt und damit auch die Anzahl der geposteten Beiträge. Die Beiträge müssen, durch Bewertungsmetriken vorsortiert, den User*innen angezeigt werden um eine Grundqualität der Seiten zu wahren. Ausgewählte Arbeiten, die sich mit der Fairness von Bewertungsmetriken beschäftigen werden vorgestellt. Agentenbasierte Modellierung findet viel in der Modellierung von Meinungsdynamiken Anwendung. Interessante Arbeiten aus diesem Bereich werden ebenfalls vorgestellt.

In dieser Arbeit wird ein agent*innenbasiertes Modell eines sozialen Mediums erstellt. Dafür ist es interessant die bisherigen verwandten Anwendungen von agent*innenbasierte Modellierung zu überblicken. Es zeigt sich, dass diese Art der Modellierung viel Anwendung in der Simulation von Meinungsbildungsprozessen und Meinungsdynamiken verwendet wird.

Viele Arbeiten beschäftigen sich mit der Fairness von Rankings in sozialen Medien. Einige Erkenntnisse und Ideen dieser Arbeiten können zur Untersuchung der Bewertungsmetriken in dieser Arbeit verwendet werden und werden in diesem Kapitel vorgestellt.

\section{agent*innenbasierte Modellierung von Meinungsdynamiken}

Um Meingungsdynamiken und Meinungsbildungsprozesse zu modellieren werden häufig agent*innenbasierte Modelle verwendet. Dabei werden Agent*innen als meinungshabende Individuen modelliert, welche andere Agent*innen in ihrer Meinung beinflussen und von anderen selbst beeinflusst werden. [Agent-Based Models for Opinion Formation: A Bibliographic Survey] ist eine bibliographische Übersicht, welche die wichtigsten Charakteristiken solcher Modelle herausarbeitet.

%Die Autor*innen von [Mixing beliefs among interacting agents] untersuchen das Verhalten von Individuen, welche eine Meinung aus einem kontinuierlichem Raum besitzen. Es wird ein Modell vorgestellt um Meinungsbildung zu simulieren.

Für eine Gruppe aus Individuen, welche alle eine eigene subjektive Meinung besitzen wird von [Reaching A consensus] ein Modell vorgeschlagen, welches die unterschiedlichen Meinungen zu einem Konsens zusammenführen kann.

In [Opinion Dynamics of Skeptical Agents] wird ein agentenbasiertes Modell vorgestellt, welcken skeptizistische Agent*innnen gegenüber anderen Meinungen beinhaltet. Es wird gezeigt, dass auch in solchen Konstellationen ein Konsens gefunden werden kann.

\section{Bewertungsmetriken und Fairness}

Einige Arbeiten beschäftigen sich mit Fairness zwischen \textit{beschützten} Gruppen, Minderheiten, und \textit{nicht geschützten} Gruppen, den Mehrheiten. Es versucht eine Benachteiligung der \textit{geschützten} Gruppen zu vermeiden. Anhand von Experimenten wird erforscht wie sich sozialer Einfluss auf individuelle Fairness in Bewertungsmetriken auswirkt. Für bekannte soziale Medien werden die Bewertungsmetriken analysiert.

%In [Fairness and Transparency in Ranking (nicht in Scopus)] stellen die Autor*innen fest, dass Fairness zwischen Gruppen hergestellt ist, wenn der Quotient aus Sichtbarkeit und Relevanz von Elementen der beiden Gruppen gleich ist. Es werden Kriterien für ein faires Ranking definiert:

%\begin{enumerate}
%\item Eine ausreichende Präsenz von Objekten aus unterschiedlichen Gruppen, speziell aus Minderheiten um statistischen Benachteiligung zu vermeiden
%\item Eine konsistente Behandlung von ähnlichen Elementen, sodass individuelle Fairness gewährleistet ist.
%\item Eine angemessene Darstellung von Objekten, speziell aus Minderheitsgruppen, um Benachteiligung durch Repräsentation zu vermeiden
%\end{enumerate}


%Die Dimensionen von Datenqualität die für User*innen auf Onlineplatformen wichtig sind werden in [Beyond Accuracy: Data Quality] beschrieben. Die 20  gefundenden Qualitätsdimensionen werden in 4 Kategorien in einem Framework zusammengefasst.

\subsection{Fairness mit Gruppen}

In [Fairness and Transparency in Ranking (nicht in Scopus)] stellen die Autor*innen fest, dass Fairness zwischen Gruppen hergestellt ist, wenn der Quotient aus Sichtbarkeit und Relevanz von Elementen der beiden Gruppen gleich ist.

[Fairness of Exposure in Rankings] und [Measuring Fairness in Ranked Output] betrachten Fairness in Rankings unter Einbeziehung von \textit{beschützten} und \textit{nicht beschützten} Objekten. Es werden Messfunktionen für Fairness eingeführt um die Benachteiligung von \textit{beschützten} Objekten systematisch auszuwerten.

Ein Algorithmus, welcher aus einer Menge von \textit{beschützten} und \textit{nicht beschützten} Objekten die relevantesten $k$ Objekte sucht wird in [FA*IR: A Fair Top-k Ranking Algorithm vorgestellt] Für diese $k$ Objekte sind nicht \textit{beschützten} Objekte nicht benachteiligt.

Gruppenbezogene Fairness wird in [Equity of Attention: Amortizing Individual Fairness in Rankings] nur als Spezialfall von individueller Fairness betrachtet. Es wird der \textit{Discounted Cumulative Gain}-Koeffizient (DCG) zur Messung der Fairness der Rankings verwendet. Die Autor*innen stellen fest, dass sich der DCG mit einer einzelnen Bewertungsmetrik nicht optimieren lässt. Sie stellen ein Optimierungsproblem auf, welche durch die Anordnung von unterschiedlichen Bewertungsmetriken Fairness \textit{amortisiert}.
%TODO das ist ganz schönes waschlappen gesabbel


\subsection{Individuelle Fairness in sozialen Medien}

In \cite{Salganik2006854} und [Measuring and Optimizing Cultural Markets] wird ein Experiment durchgeführt in dem Testpersonen unbekannte Lieder bewerten. Die Testpersonen werden in zwei Gruppen aufgeteilt. Eine \textit{unabhängige} Gruppe und eine unter \textit{sozialem Einfluss}. die \textit{unabhängige} Gruppe erhält keine weiteren Informationen, alle User*innen kriegen eine zufällige Liste der Lieder angezeigt und können diese downloaden. Die Gruppe unter \textit{sozialem Einfluss} wird zusätzlich in acht Welten unterteilt, jeder User kriegt eine Liste mit Liedern nach der Downloadzahl sortiert angezeigt. Die Downloadzahlen werden angezeigt. Die Autoren zeigen, dass unter sozialem Einfluss Ungleichheit und Unvorhersehbarkeit der Popularität von Objekten besteht.

In [Leveraging Position Bias to Improve Peer Recommendation] wurde ein Experiment durchgeführt, bei dem Proband*innen Posts anderen Proband*innen empfehlen können, welche schließlich nach 5 unterschiedlichen Bewertungsmetriken gerankt werden: 
\begin{enumerate}
	\item Zufall: in zufälliger Reihenfolge
	\item Popularität: nach der Anzahl der Empfehlungen sortiert
	\item Aktitvität: nach der Zeit der letzten Empfehlung sortiert
	\item Fixierung: stets in gleicher Reihenfolge
	\item Fixierung invertiert: in umgekehrter fixierter Reihenfolge
\end{enumerate}

Die Bewertungsmetriken werden mit dem \textit{Gini}-Koeffizienten ausgewertet. Dabei schneidet die Zufallsmetrik, gefolgt von der Aktivitätsmetrik am besten, die fixierten Metriken am schlechtesten ab.


In [How Not To Sort by Average Rating] wird der \textit{Wilson}-Score zur Auswertung von Up- und Downvotes vorgeschlagen. Es wird gezeigt, weshalb andere Methoden schlechter geeignet sind. Der \textit{Wilson}-Score findet im Reddit-"Best"-Kommentarranking Anwendung.

Für Rankings mit den Bewertungsmöglichkeiten "Daumen hoch" und "Daumen runter" stellt [How to Count Thumb-Ups and Thumb-Downs:User-Rating based Ranking of Itemsfrom an Axiomatic Perspective] intuitive Axiome vor, welche ein Auswertung der "Daumen" erfüllen sollte. Bekannte Methoden wie der \textit{Wilson}-Score werden auf die Axiome geprüft. % und eine Methode vorgestellt, welche die Axiome erfüllt.

Die Autor*innen stellen in [Social dynamics of Digg] und [Using an model of social dynamics to predict popularity of news] ein stochastisches Modell auf um die Popularität von Beiträgen auf der Platform \textit{Digg} vorherzusagen. Modellparameter, wie die Aktivitätsverteilung von User*innen wird durch einen Datensatz von \textit{Digg} gefittet. Das Modell wird validiert und erfasst die Hauptkomponenten der Bewertungsdynamiken von \textit{Digg}

[Stoddard] untersucht die Korrelation zwischen Qualität von Posts und die Anzahl und Art der Bewertungen auf Hacker News und Reddit. Die Qualität eines Posts wird als die Bewertung beschrieben, welche ein Post unter absolut fairen Bedingungen erhalten würde. Sie entwickeln eine Methode um die Qualität von Posts auf den Platformen abzuschätzen.

In [Description and Prediction of Slashdot Actitvity] wird versucht die Aktitvität die ein Post auf der Platform \textit{Slashdot} ingesamt erzeugt auf Grundlage der erzeugten Aktivität in den ersten Minuten/Stunden nach Veröffentlichung des Posts. 

Muchnik findet in \cite{Muchnik2013647} heraus, dass sozialer Einfluss Bewertungsdynamiken verzerrt und zur Bildung von Bewertungsblasen führen kann.
Negativer sozialer Einfluss kann durch die Gruppenintelligenz wieder ausgeglichen werden.

In [How Public Opinion Forms] wird gezeigt, dass die Popularität von Beiträgen Einfluss auf das Bewertungsverhalten hat. So sammeln populäre, gute bewertete Beiträge, zunehmend auch schlechte Bewertungen.

In [Towards Quality Discourse in Online News Comments]  werden Kommentarsystem von Nachrichtenagenturen untersucht und die Wirkung von Beiträgen mit niedriger Qualität auf Nutzer*innen und Journalist*innen. Es wird gezeigt, wie die individuelle Lesemotivation Einfluss auf die Qualitätswahrnehmung hat. Um die Qualität zu verbessern werden unter anderem Moderations- und Markierungsmethoden vorgeschlagen.

Luu schlägt in [Randomize HN] vor etwas Random Noise in das Ranking von Hacker News einzufügen, um die scharfe Aufmerksamkeitskante zwischen Posts die auf der Startseite landen und denen auf  den hinteren Seiten auszuglätten.

Die Autoren von [Ranking with Fairness Constraints] argumentieren, dass viele Bewertungsmetriken die Diversität verringern und Stereotypen reproduzieren, die Bewertungsmetriken jedoch nicht darauf geprüft werden. Es wird ein Algorithmus vorgeschlagen, welcher ein Optimierung zur Bestimmung eines Objektrankings unter bestimmten Fairnessbedingungen vornimmt.

%Diese Arbeit greift Ideen und Kritik zu bestehenden Bewertungsmetriken und deren Evaluation auf. Es wird ein Framework entwickelt, durch welches Bewertungsmetriken durch ein agentenbasiertes Modell simuliert und verglichen werden können. 


%Eine Methode um Beiträge in Q\&A Foren nach Qualität zu klassifizieren wird in [Understanding and Classifiying the Quality of Technical Forum Questions] vorgeschlagen. Die Methode verwendet \textit{Decision Trees}. Sie wird an einem Datensatz von Stack Overflow validiert.



%- Distributed Moderation Systems: An Exploration of Their Utility and the Social Implications of Their Widespread Adoption

%Mill untersucht die Aktivität von Usern auf Reddit und stellt Zusammenhänge mit dem Votingverhalten fest.

%(Diese Mega ausführliche Doktorarbeit)

%Definiert Qualität als die Anzahl von Votes, die ein Artikel erhalten würde, wenn die Artikel in zufälliger Reihenfolge ohne sozialen Einfluss, wie der aktuelle Score angezeigt würden. Sie finden heraus, dass in Reddit und Hacker News die Qualität von Posts und die Anzahl an Votes korreliert sind, Mit einem Spearman-Korrelationskoeffizienten von 0.8 bei Hacker News und einem Koeffizienten zwischen 0.54 und 0.75 für unterschiedliche Subreddits
