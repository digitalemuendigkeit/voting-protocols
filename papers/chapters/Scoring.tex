\chapter{Models und Bewertungsmetriken}

Unterschiedliche Scoringprotokolle werden über Modelle und Scoringfunktionen implementiert

- Standard Model



- Downvote model


- random model 

die Scores werden mit etwas random noise bedacht

user können 


-  Vote evaluation

	Es können unterschiedliche Ansätze verfolgt werden um die Hoch und Runterbewertungen gegeinander abzuwiegen. 
	Wie in [Online Paper über Wilson] erläutert wird, wird durch den Quotient und aus Upvotes durch Gesamtvotes sowie die Differenz aus Upvotes und Downvotes in bestimmten Konstellationen unintuitiv bewertet. Da jedoch das Reddit Hot ranking die Differenz als Metrik verwendet, wird diese weiterhin implementiert und verwendet
	
- Differenz
	
	upvotes - downvotes
	
\begin{equation}
diff(post_{i}) = u_{i} - d_{i}
\end{equation}
- Wilson Score

	Eine von [] vorgeschlagene Metrik ist die untere Grenze des Wilson-Konfidenzintervall für die Erfolgswahrscheinlichkeit der Binomialverteilung für den Parameter $p$. Dabei ist $n$ die Gesamtanzahl an abgegeben Votes an einen Post, die Stichprobengröße, und $k$ die Anzahl an positiver Bewertungen eines Postes, die Anzahl an Erfolgen in der Stichprobe.
	
\begin{align}
 c = \Phi^{-1}(1 - \frac{\alpha}{2}) \\
 n = u_{i} + d_{i} \\
 \hat{p_{i}} = \frac{u_{i}}{n}  \\
 wilson(p_{i}) = \frac{1}{1+\frac{c^2}{n}}*(\hat{p_{i}} + \frac{c^2}{2n} - c* \sqrt{\frac{\hat{p_{i}}*(1 - \hat{p_{i}})}{n} + \frac{c^2}{4n^2}})
\end{align}
	
	

\section{Bewertungsmetriken}

Bewertungsmetriken berechnen den Score $S_{p,t}$ für einen Post unter Betrachtung der Postparameter $x_{p,t}$ eines Postes $p$ zum Zeitpunkt $t$ . Die vorgestellten Bewertungsmetriken von Hacker News und Reddit werden im produktiven Betrieb verwendet. Es werden weitere Metriken eingeführt um diese miteinander zu vergleichen. 
	
\section{Hacker News}
\label{seqHackerNews}

Die Bewertungsmetrik von Hacker News in \ref{HackerNews} wird beschrieben durch den Quotienten aus Upvotes eines Postes und dem Alter eines Postes in Stunden. Den Upvotes $u$ wird $1$ abgezogen, um den Upvote der Urheber*in des Posts auszugleichen. $G$ ist die \textit{Gravität} und beschreibt, wie schnell Posts mit der Zeit an Score verlieren. In der Default Hacker News Bewertungsmetrik gilt $G = 1.8$.

\begin{equation}
\label{HackerNews}
S_{p,t} = \frac{u_{p,t} - 1}{(age_{p,t} + 2)^{G}}
\end{equation}


\section{Reddit Hot Ranking}

Es werden in \ref{RedditHotT} die Sekunden $s$ die zwischen dem Veröffentlichungszeitpunkt $r$ eines Posts und dem Zeitpunkt $E$ am 8.12.2005 um 07:46:43 liegen berechnet.

\begin{equation}
\label{RedditHotT}
s_{p} = r_{p} - E  
\end{equation}

Im Reddit Hot Ranking fließen die Up- und Downvotes eines Postes als Differenz $d$ in \ref{RedditHotD} ein.  

\begin{equation}
\label{RedditHotD}
d_{p,t} = diff(p_t)
\end{equation}
\\
Der Parameter wird $z$ wird in \ref{RedditHotZ} auf das Maximum des Betrages von $d$ und 1 gesetzt.

\begin{equation}
\label{RedditHotZ}
z_{p,t}  = \begin{cases}
|d_{p,t}| &\text{falls $|d_{p,t}| \geq 1$}\\
1 &\text{sonst}
\end{cases}
\end{equation}

In der Bewertungsmetrik des Reddit Hot Ranking in \ref{RedditHot} fließt $z$ logarithmisch unter Beachtung des Vorzeichens von $d$ ein. $t$ wird skaliert addiert.

Somit werden Posts mit fortschreitender Zeit nicht schlechter bewertet, wie bei der Bewertungsmetrik von Hacker News. Neuere Posts erhalten durch das Ansteigen von $t$ eine höhere Bewertung, bei gleichem $d$, als ältere Posts.  


\begin{equation}
\label{RedditHot}
S_{p,t} = sign(d_{p,t}) * log_{10}(z_{p,t}) + \frac{s_{p}}{4500}
\end{equation}



\section{View Bewertungsmetrik}

Die View Bewertungsmetrik basiert auf der in Kapitel \ref{seqHackerNews} beschriebenen Metrik. Es fließt die Anzahl der Betrachtungen des Posts $v$ mit ein. Daraus ergibt sich die Bewertungsmetrik in \ref{ViewScoring}:

\begin{equation}
\label{ViewScoring}
S_{p,t} = \frac{\frac{u_{p,t} - 1}{v_{p,t} + 1}}{(age_{p,t} + 2)^{G}}
\end{equation}


\section{Aktivität}

Die Bewertung des Posts wird der Bewertung der letzten Iteration verrechnet und durch das Alter skaliert. Mit fortschreitender Zeit verlieren Posts in dieser Metrik an Score, so ergibt sich in \ref{Activation}:

% Möglicherweise auf S_{i,j} als i_ter Post und j_ter Schritt

\begin{equation}
\label{Activation}
S_{p,t} = \frac{eval(p_{t}) - S_{p,t-1}}{(age_{p,t} + 2)^{G}}
\end{equation}

	