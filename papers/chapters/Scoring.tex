\chapter{Bewertungsmetriken}
\label{chapter:bewertungsmetriken}

Bewertungsmetriken werden verwendet um die große Menge an Posts zu strukturieren. Den Posts wird anhand ihrer aktuellen Parameter, wie die Art und Anzahl der Bewertungen, oder dem Veröffntlichungszeitpunkt ein Score zugeordnet. 
Sortiert nach diesem Score werden die Posts auf der Plattform angeordnet.

Eine Bewertungsmetrik soll fair sein. Eine faire Bewertungsmetrik $S$ zeichnet sich durch folgende Punkte aus:

\begin{itemize}
	\item Für zwei Posts $P_1$ und $P_2$ mit $a_{P_1} = a_{P_2}$ und ${q_{P_1}}_{ges} > {q_{P_2}}_{ges}$ ist $S(P_1) > S(P_2)$ in jedem Iterationsschritt
	\item Für jeden Post $P_i$ gilt $w_{P_i} > 1$, er wurde von mindestens eine*r User*in wahrgenommen
\end{itemize}

\section{Bewertungstransformation}
\label{voteevaluation}

Die Bewertungsvektoren liegen im $N_B$-dimensionalen Raum vor. Innerhalb von Bewertungsmetriken werden die $N_B$ dimensionalen Bewertungsvektren von Posts auf einen Skalar transformiert. Dies geschieht mit einer Bewertungstransformationsfunktion $v: \mathbb{R}^{n} \rightarrow \mathbb{R}$.


Für den zweidimensionalen Fall mit Up- und Downvotes ($u_P$,$d_P$) werden im folgenden einige Bewertungstransformationen vorgestellt.

\subsection{Differenz}

Es wird die Differenz aus Up- und Downvotes zu berechnet, so ergibt sich in Formel \ref{diff}:

\begin{equation}
\label{diff}
v_{diff}(b_P) = u_P - d_P
\end{equation}

Die Differenz wird im Reddit Hot Ranking verwendet.

\subsection{Anteil}

Berechnet wird der Anteil der Upvotes zur Gesamtzahl der Votes:

\begin{equation}
\label{anteil}
v_{anteil}(b_P) = \frac{u_P}{u_P + d_P}
\end{equation}


\subsection{Wilson Score}

Wie in [Online Paper über Wilson] erläutert wird, ist die Verwerdung der beiden vorherig vorgestellten Metriken in bestimmten Konstellationen von Up- und Downvotes unintuitiv. Es wird  die untere Grenze des Wilson-Konfidenzintervall für die Erfolgswahrscheinlichkeit der Binomialverteilung für den Parameter $p$ als Metrik vorgeschlagen. Dabei ist $n$ die Gesamtanzahl an abgegeben Votes an einen Post, die Stichprobengröße, und $k$ die Anzahl an positiver Bewertungen eines Postes, die Anzahl an Erfolgen in der Stichprobe.
	
Mit der Gesamtzahl der Bewertungen eines Posts $n_P = u_P + d_P$, dem Punktschätzer $\hat{p}_P = u_P + n_P$ und dem Quantil $c_{\alpha}$ der Normalverteilung zum Irrtumsniveau $\alpha$ ergibt sich in Formel \ref{wilson} der Wilson Score:

\begin{equation}
\label{wilson}
 v_{wilson}(b_P) = \frac{1}{1+\frac{c_{\alpha}^2}{n_P}}(\hat{p}_P + \frac{c_{\alpha}^2}{2n_P} - c \sqrt{\frac{\hat{p}_P(1 - \hat{p}_P)}{n_P} + \frac{c_{\alpha}^2}{4n_P^2}})
\end{equation}
	
\section{Bewertungsmetriken}

Bewertungsmetriken berechnen den Score $S_{P,t}$ für einen Post unter Betrachtung der Postparameter eines Postes $P$ zum Modellschritt $t$.
	
\subsection{Hacker News Ranking}
\label{seqHackerNews}

%TODO nochmal über die Notation S_{P,t} klar werden, vielleicht auch S_t(P) ??
Hacker News verwendet eine Metrik, in der die Upvotes $u_{P}$ eines Postes ins Verhältnis zum Alter $a_{P}$ in Stunden zum Zeitpunkt $t$ setzt. Somit besitzt der Bewertungsraum die Dimension 1 .Das Alter wird mit der Gravitätskonstanste $G = 1.8$ potenziert. Die Bewertungstransformation lautet $v(b_P) = u_P - 1$ so ergibt sich die Hacker News Bewertungsmetrik in Formel \ref{HackerNews}:
 
\begin{equation}
\label{HackerNews}
S_{P,t} = \frac{u_{P,t} - 1}{(a_{P,t} + 2)^{G}}
\end{equation}

\section{Verallgemeinertes Hacker News Ranking}
\label{seqvHackerNews}
Das verallgemeinerte Hacker News Ranking verwendet eine beliebige Bewertungstransforamiton $v$ und lässt sich somit auch auf höhere Bewertungsräume anwenden:

\begin{equation}
\label{vHackerNews}
S_{P,t} = \frac{v(b_{P,t})}{(a_{P,t} + 2)^{G}}
\end{equation}


\section{Reddit Hot Ranking}

% TODO Karma wird nicht einberechnet

Im Reddit Hot Ranking fließt die evaluierte Voteanzahl $d_P$ eines Postes logarithmisch, und der Zeitpunkt der Veröffentlichung $r_P$ einfach ein. 

Um den Veröffentlichungszeitpunkt $e_P$ zu messen, wird die Differenz in Sekunden von diesem zum Zeitpunkt $E$ am 8.12.2005 um 07:46:43 in Formel \ref{RedditHotT} berechnet.

\begin{equation}
\label{RedditHotT}
e_{P} = r_{P} - E  
\end{equation}

Die Default Bewertungstransformation des Reddit Hot Ranking ist die Differenz aus Up- und Downvotes:

\begin{equation}
\label{RedditHotD}
v(b_{P}) = v_{diff}(b_{P})
\end{equation}
\\
Der Parameter $z$ wird in \ref{RedditHotZ} auf das Maximum des Betrages von $v(b_{P,t})$ aus Formel \ref{RedditHotD} und 1 gesetzt.

\begin{equation}
\label{RedditHotZ}
z_{P,t}  = \begin{cases}
|v(b_{P,t})| &\text{falls $|v(b_{P,t})| \geq 1$}\\
1 &\text{sonst}
\end{cases}
\end{equation}

Es ergibt sich die Bewertungsmetrik des Reddit Hot Rankings in Formel \ref{RedditHot}:

\begin{equation}
\label{RedditHot}
S_{P,t} = sign(v(b_{P,t})) * log_{10}(z_{P,t}) + \frac{e_{P}}{4500}
\end{equation}

Somit werden Posts mit fortschreitender Zeit nicht schlechter bewertet, wie bei der Bewertungsmetrik von Hacker News. Neuere Posts erhalten durch das Ansteigen von $e_P$ eine höhere Bewertung, bei gleichem $v(b_P)$, als ältere Posts. 
 

\section{View Bewertungsmetrik}

Die View Bewertungsmetrik basiert auf der in Kapitel \ref{seqvHackerNews} beschriebenen Metrik. Es fließt die Anzahl der Betrachtungen des Posts $w$ mit ein. Daraus ergibt sich die Bewertungsmetrik in \ref{ViewScoring}:

\begin{equation}
\label{ViewScoring}
S_{p,t} = \frac{\frac{b_{P,t}}{w_{p,t} + 1}}{(a_{p,t} + 2)^{G}}
\end{equation}


% TODO View Exp und View activation noch hinzufügen?? Vielleicht nicht ist auch so schon viel


\section{Aktivität}

Die Bewertung des Posts wird der Bewertung der letzten Iteration verrechnet und durch das Alter skaliert. Mit fortschreitender Zeit verlieren Posts in dieser Metrik an Score, so ergibt sich in \ref{Activation}:

\begin{equation}
\label{Activation}
S_{p,t} = \frac{v(b_{P,t})) - S_{p,t-1}}{(a_{p,t} + 2)^{G}}
\end{equation}


\section{Zufallsbewertung}

Nachdem Posts durch die Bewertungsmetrik bewertet wurden, kann die Bewertung durch Zufall verunreinigt werden um den in \cite{Luu} vorgeschlagenen Lärm zur Bewertung hinzuzufügen. In dieser Arbeit wurden zwei unterschiedliche Ansätze zur zufälligen Verunreinigung gewählt.

\subsection{Mittelwertabweichung}

Sei $\mu_s$ der Mittelwert der Scores aller Posts. Für einen Post $p$ mit Score $s_p$ wird die Verunreinigung $d$ als Zufallszahl im Intervall $d_{\mu,p} \in [-|\mu_s - s_p|,|\mu_s - s_p|]$ definiert. So ergibt sich für den verunreinigten Score $\tilde{s}_p$ des Post in Formel \ref{mean_deviation}:

\begin{equation}
\label{mean_deviation}
\tilde{s}_{\mu,p} =  s_p + d_{\mu,p}
\end{equation} 

\subsection{Standardabweichungabweichung}

Für die Standardabweichung der Scores aller Posts $\sigma_s$ sei die Verunreinigung eine Zufallszahl im Intervall $d_{\sigma,p} \in [-\sigma_s,\sigma_s]$, sodass sich der verunreinigte Score eines Postes in Formel \ref{std_deviation} ergibt:

\begin{equation}
\label{std_deviation}
\tilde{s}_{\sigma,P} = s_P + d_{\sigma,P}
\end{equation}

	