\chapter{Einleitung}

\section{Motivation}

In den letzten Jahrzehnten sind soziale Medien zu einem wichtigen Teil in der Kommunikation herangewachsen. Im Gegensatz zu konventionellen Medien können Nutzer*innen von sozialen Medien selbst Inhalte veröffentlichen. Dies führt zu einer Menge an Information, die kein Mensch überblicken oder verstehen kann. Es werden Algorithmen verwendet, welche die Beiträge filtern und den Nutzer*innen diese anzeigen, welche sie wahrscheinlich interessant finden. 

Viele soziale Medien und Onlineplattformen verwenden \textit{Votingsysteme} um Nutzer*innen die Möglichkeit zu geben iher Meinung über den Beitrag zu äußern. Hacker News\footnote{\texttt{https://news.ycombinator.com}} erlaubt User*innen Posts nur positiv, mit Upvotes, zu bewerten. Auf Reddit\footnote{\texttt{htttps://www.reddit.com}} ist es ebenfalls möglich Posts mit Downvotes runterzuwerten. Onlineshops wie TheCubicle\footnote{\texttt{https://www.thecubicle.com}} verwenden ein 1 bis 5 Sterne System um Artikel zu bewerten.

Es können Empfehlungssysteme verwendet werden, welche Nutzer*innen die Beiträge personalisiert anzeigen, welche bereits andere Nutzer*innen mit ähnlichen Interessen als gut bewertet haben. Während User*innen aus ihrer Perspektive interessante Posts erhalten hat diese Methode einige negative Effekte. Die Anwendung führt zur Bildung von Filterblasen. User*innen interagieren nur noch mit Beiträgen, welche sie positiv wahrnehmen. Nutzer*innen welche sich zum Beispiel in der Filterblase eines politischen Extrems befinden erhalten so keine kritischen Beiträge mehr zu ihrer eigenen Meinung wodurch sich diese verstärkt. 

Beiträge welche bereits viel Aufmerksamkeit erhalten haben, werden weiteren Nutzer*innen angezeigt und erhalten dadurch weitere Aufmerksamkeit und Interaktion. Beiträge hingegen, welche wenig Aufmerksamkeit erhalten, werden durch das Empfehlungssystem nicht weiter als interessant erachtet. Diese Effekt wird \textit{Matthäus-Effekt} bezeichnet.

Filterblasen können vermieden werden, wenn allen Nutzer*innen die gleichen Beiträge angezeigt werden. Die Beiträge müssen trotzdem anhand des Interesses für Nutzer*innen gefiltert und sortiert werden. Dazu werden \textit{Bewertungsmetriken} verwendet, welche die Posts anhand ihrer Interaktion, zum Beispiel unter Einbeziehung ihrer Upvotes und des Veröffentlichungszeitpunkt bewerten und anordnen.

Auch Bewertungsmetriken sind Matthäus-Effekten ausgesetzt. Im Gegensatz zu Empfehlungssystemen basieren diese meist nicht auf Methoden des maschinellen Lernens, sondern auf einfachen Algorithmen welche leicht austauschbar sind. 

Es ist wünschenswert eine Bewertungsmetrik zu finden, welche die Matthäus-Effekte minimiert. Aufgrund ihrer einfachen Struktur können sie leicht simuliert und verglichen werden.

Es ist nicht ausgeschlossen, dass die Art des Votingssystems Einfluss auf die Matthäus-Effekte hat. Das Votingsystem sollte auf der Suche nach einer optimalen Bewertungsmetrik mit einbezogen werden.

In dieser Arbeit wird ein agent*innen basiertes Modell von Votingsystemen in sozialen Medien entworfen um unterschiedliche Bewertungsmetriken zur vergleichen.

%Wie in [Description and Prediction of Slashdot Activity] gezeigt wird ist es entscheidend wie viel Aktivität ein Beitrag in einer kurzen Zeitspanne erzeugen kann. Ist es viel Aktivität wird der Beitrag insgesamt viel Aufmerksamkeit erhalten, ist es jedoch wenig Aktivität wird der Beitrag auch insgesamt wenig Aufmerksamkeit erfahren.

% TODO Grafik nach Slashdot Activtiy ranking nachempfinden


%Keine Möglichkeit ein Experiment durchzuführen wie in dieser Arbeit von Salganik oder Lerman

\section{Agent*innenbasierte Modellierung}

In der agent*innenbasierten Modellierung werden die Aktionen und Interaktionen einzelner Entitäten, den Agent*innen, simuliert. Auch die Umwelt in der die Agent*innen interagieren wird modelliert. Mit Agent*innen können unterschiedliche Dinge modelliert werden. So sind Agent*innen nicht auf die Modellierung von Menschen beschränkt. Durch Agent*innen können in einer Simulation eines Vogelschwarms die Vögel modelliert werden. Für die Modellierung eines Waldbrandes würden die Agent*innen Bäume repräsentieren. Einzelne Agent*innen können sich dabei in ihren Eigenschaften sehr unterschiedlich sein. Werden Menschen durch die Agent*innen modelliert können diese können zum Beispiel extrovertiert oder introvertiert sein, eigennützig oder im Sinn der Gruppe handeln.

Agent*innenbasierte Modelle sind weder absolut realistisch noch vollständig. Durch sie wird eine vereinfachte Realität modelliert. Die Agent*innen können zwar beliebig komplex parametrisiert werden. Dennoch werden meist sehr einfache Verhaltensregeln der Agent*innen definiert, denn durch die Interaktion entstehen bereits hier komplexe Systeme, welche die Realität scheinbar ausreichend approximieren.

Kleine Veränderungen am Verhalten der Agent*innen können bereits große Veränderungen des Simulationsergebnisses hervorrufen.In der Simulation treffen die Agent*innen individuell unterschiedliche Entscheidungen, welche durch die Verhaltensregeln vorgegeben sind und durch die Wahrnehmung der Agent*innen der Umwelt beeinflusst wird.

Die Agent*innen können in einem Netzwerk angeordnet sein, sodass sie sich gegenseitig in ihrem Verhalten beeinflussen, sie können jedoch auch nur mit dem System interagieren.
%Wie [Agents.jl] beschreibt können viele komplexe Systeme nicht vollständig durch herkömmliche mathematische Methoden beschrieben werden. Komplexe Systeme hängen von dem Verhalten der Elemente, der Agenten, des Systems ab. Kleine Änderungen am Verhalten der Agenten können große Veränderungen des Gesamtsystems hervorrufen. Durch agentenbasierte Modellierung können nicht lineare Modelle beschrieben.

%Agenten können in einem Raum angeordnet werden, sodass sie mit ihren Nachbarn interagieren, wie im Modell von [Shelling], oder als Netzwerk indem jeder Agent mit jedem Agenten interagieren kann. Möglich ist auch, dass die Agenten nicht untereinander, nur mit dem System, interagieren.

Die Popularität steigt stetig, agent*innenbasierte Modelle sind beliebt um unter anderem biologische, ökonomische und soziale Systeme, wie soziale Medien zu modellieren. Auch in dieser Arbeit wird die agent*inenbasierten Modellierung zur Simulation einer Social Media Plattform verwendet.

\section{Die modellierte Kommunikationsplattform}

User*innen können Posts veröffentlichen. Diese sind für alle anderen User*innen sichtbar und können bewertet werden. Die Posts werden nach einer Bewertungsmetrik unter Betrachtung der Postparameter, wie die Anzahl und Art der Bewertungen, Betrachtungen und Zeitpunkt der Veröffentlichung bewertet. Auf der Startseite der Plattform werden die Posts absteigend nach ihrer Bewertung sortiert in einer vertikalen Liste angezeigt. Je weiter ein*e User*in auf der Seite herunterscrollt, desto mehr Posts werden angezeigt. Die Posts sind nicht auf Seiten aufgeteilt, sondern befinden sich in einer kontinuierlichen Liste. Die Liste der Posts ist für alle User*innen, die die Plattform zum gleichen Zeitpunkt besuchen identisch und nicht personalisiert.

Im Laufe der Modellierung erhält die Plattform keine neue User*innen, es werden jedoch neue Posts erstellt und hinzugefügt.

